\chapter{Mathematical Induction}

% ===========================================================
\section{Sum and Product Notation}
% ===========================================================

In the last note, we ended up working with polynomials with an arbitrary number of terms, relying on the following notation.

\begin{align*}
	p(x) = \underbrace{a_n x^n + a_{n - 1} x^{n - 1} + \dots + a_1 x + a_0}_\text{$n - 1$ terms}
\end{align*}

While this was perfectly rigorous, it relies on the reader being able to notice the pattern among terms in the summation. It may be less ambiguous to explicitly state this pattern, for which we introduce the following notation.

\begin{align*}
	\sum_{i = a}^b f(i) &= f(a) + f(a + 1) + \dots + f(b - 1) + f(b) \\
	\prod_{i = a}^b f(i) &= f(a) f(a + 1) \dots f(b - 1) f(b)
\end{align*}

Using this notation, we would express a generic $n$ degree polynomial with coefficients $\{ a_n \dots a_0 \}$ as the following.

\begin{align*}
	p(x) &= a_n x^n + a_{n - 1} x^{n - 1} + \dots + a_1 x + a_0 \\
	&= \sum_{i = 0}^n a_i x^i
\end{align*}

Notice that both of these are common computational processes in computer science which can be expressed as with the following pseudo-code.

\vspace{\baselineskip}
\begin{lstlisting}
	def sum(f, a, b):
		sum = 0
		for i in [a, ..., b]:
			sum = sum + f(i)
		return sum
		
	def prod(f, a, b):
		prod = 1
		for i in [a, ..., b]:
			prod = prod * f(i)
		return prod
\end{lstlisting}

The following properties of summation and product naturally extend from its definition, and should intuitively be true, though writing out a proof should not be too difficult.

\begin{align*}
	\sum_{i = a}^b cf(i) &= c \sum_{i = a}^b f(n) \\
	\sum_{i = a}^b (f(i) + c) &= \underbrace{(b - a + 1)}_\text{number of terms}c + \sum_{i = a}^b f(n) \\
\end{align*}

There is an analogous equivalence for a product, though this time the number we we pull out get raised to the power of the number of terms.

\begin{align*}
	\prod_{i = a}^b cf(i) &= c^{b - a + 1} \prod_{i = a}^b f(n) \\
\end{align*}

These summations (and products) appear frequently in mathematics, especially in calculus where the number of terms goes to infinity, so over the course of this chapter, we help develop tools to help simplify then more useful formula. As an example, consider the sum of the first $n$ natural numbers.

\vspace{\baselineskip}
\begin{theorem}
	For all natural numbers $n$, the following equation holds.
	\begin{align*}
		\sum_{i = 1}^n i &= 1 + 2 + \dots + n \\
		&= \frac{n(n + 1)}{2}
	\end{align*}
\end{theorem}
\begin{proof}
	We offer a direct proof of this fact using a trick; first denote the sum itself as $S$, so we have the expression
	
	\begin{align*}
		S = 1 + 2 + 3 + \dots + (n - 1) + n.
	\end{align*}
	
	Now we sum the sequence the with itself in reverse order, in essence adding two equations.
	
	\begin{align*}
		\begin{array}{cccccccccccc}
			S = & 1 & + & 2 & + & \dots & + & (n - 1) & + & n \\
			S = & n & + & (n - 1) & + & \dots & + & 2 & + & 1 \\ \hline
			2S = & (n + 1) & + & (n + 1) & + & \dots & + & (n + 1) & + & (n + 1) \\
		\end{array}
	\end{align*}
	
	Notice in the bottom sum, each of the $n$ terms is $(n + 1)$. Therefore we can simplify the expression even further to get
	
	\begin{align}
		2S = n(n + 1) \text{ or} \\
		S = \frac{n(n + 1)}{2}
	\end{align}
	
	This is exactly what we need to show, hence we have derived the formula.
\end{proof}
\vspace{\baselineskip}

Notice the simplified expression is more useful to us, since it more clearly expresses the relationship between $n$ and the sum. More specifically, from this formula we can tell that the sum is actually a well-defined polynomial in $n$, a fact which is obfuscated by the summation. However in the above proof we used a trick, and it is not clear how we would approach proving formulae like this more generally.

% ===========================================================
\section{The Naturals Numbers}
% ===========================================================

While most of our proofs until how have been focused on the real numbers and integers, we now turn our attention to the \emph{natural numbers.} To develop another fundamental proof technique for working with these numbers, we need a more precise definition of the natural numbers, which we currently refer to with the set

\begin{align*}
	\mathbb{N} = \{ 1, 2, 3, \dots \}.
\end{align*}

To avoid the ambiguity of the ellipses, we can define the natural numbers ($\mathbb{N}$) as the intersection of all subsets $A$ of the real numbers with the following properties

\vspace{\baselineskip}
\begin{enumerate}[\hspace{\baselineskip}i.]
	\item $1 \in A$
	\item if $n \in A$, then $s(n) \in A$ where $s$ is the successor function which represents $s(n) = n + 1$.
\end{enumerate}
\vspace{\baselineskip}

Notice this is consistent with our intuitive definition of the natural numbers. If $A$ obeys these two definitions, then we know 1 must be in $A$, along with all number in the sequence $1, 2, 3, \dots$

\begin{align*}
	\{ 1, 2, 3, \dots \} \subseteq A
\end{align*}

Furthermore, since this set itself satisfies these conditions, we know that it is an example of one such set $A$. Therefore the intersection of all such sets is indeed what we expected,

\begin{align*}
	\mathbb{N} = \{ 1, 2, 3, \dots \}.
\end{align*}

Taking the intersection of all such sets $A$ is essentially equivalent to taking the ``smallest'' set which satisfies these properties; otherwise, notice that the real numbers themselves also satisfy both properties. With this definition, we have a more systematic way to understand the natural numbers which yields a new proof technique.

\vspace{\baselineskip}
\begin{theorem}[Principle of Induction]
	Let $P(n)$ be a predicate defined for natural numbers $n \in \mathbb{N}$. If the following two statements hold,
	
	\vspace{\baselineskip}
	\begin{enumerate}[\hspace{\baselineskip}i.]
		\item $P(1)$ is true
		\item if $P(k)$ is true, then $P(k + 1)$ is true
	\end{enumerate}
	\vspace{\baselineskip}

	then $P(n)$ is true for all natural numbers $n \in \mathbb{N}$.
\end{theorem}
\begin{proof}
	We consider the set $A$ for which this predicate is true and show that this is equal to the set of natural numbers.
	
	\begin{align*}
		A = \{ n : P(n) \text{ is true} \}
	\end{align*}
	
	First notice that since $P(n)$ is defined for natural numbers only, hence immediately we know $A \subseteq \mathbb{N}$. Now we know that
	
	\vspace{\baselineskip}
	\begin{enumerate}[\hspace{\baselineskip}i.]
		\item $1 \in A$ because $P(1)$ is true, and
		\item if $P(k)$ is true, then $P(k + 1)$ is true, so if $k \in A$ then $(k + 1) \in A$.
	\end{enumerate}
	\vspace{\baselineskip}

	Therefore we have shown that $A$ is one of the sets used in the above definition. Since $\mathbb{N}$ is the intersection of all possible such sets, we know that it must be contained in $S$ so that
	
	\begin{align*}
		\mathbb{N} \subseteq A
	\end{align*}
	
	Since we have showed $A$ is both a subset and a superset of $\mathbb{N}$, this means $A = \mathbb{N}$. Since $A$ represents the set of number for which $P(n)$ is true, this means $P(n)$ is true for all natural numbers $n$.
\end{proof}
\vspace{\baselineskip}

This means that we now have a proof technique which can be applied to proving statements which are true for over the natural numbers, namely by showing that the statements holds for $n = 1$ and that for all $k$, if $P(k)$ is true then it implies $P(k + 1)$. This can be precisely expressed with the following logical equivalence, which conveys the same idea.

\begin{align*}
	(\forall n \in \mathbb{N})P(n) \equiv P(1) \land (\forall k \in \mathbb{N})[P(k) \implies P(k + 1)]
\end{align*}

% ===========================================================
\section{Inductive Proofs}
% ===========================================================

Through the above manipulation, we have discovered a fundamentally different way to prove universally quantified statements, specifically those which are true over the natural numbers, by exploiting the definition of the natural numbers. The logical equivalence

\begin{align*}
	(\forall n \in \mathbb{N})P(n) \equiv P(1) \land (\forall k \in \mathbb{N})[P(k) \implies P(k + 1)],
\end{align*}

shows exactly what is required, but we can unwrap this. Proving some statement $(\forall n \in \mathbb{N})P(n)$ using this tool is called a \emph{proof by induction.} Since the right-hand side of the expression is a conjunction, we need to show both sides are true, meaning a proof by induction has two separate parts.

\vspace{\baselineskip}
\begin{enumerate}[\hspace{\baselineskip}i.]
	\item \emph{Base Case.} Show $P(1)$ is true.
	\item \emph{Induction Step.} Assume $P(k)$ is true for any natural number $k$, and use this to show $P(k + 1)$ must be true.
\end{enumerate}
\vspace{\baselineskip}

The validity of this proof technique follows directly from the principle of mathematical induction. We can use this the approach the problem of proving equalities like the following systematically.

\begin{align*}
	\sum_{i = 1}^{n} i = \frac{n(n + 1)}{2}
\end{align*}

First, let us define the statement we are trying to prove more explicitly. If we denote the above expression with the predicate $P(n)$, we want to show that if the free variable $n$ is a natural number, then the statement is true, or

\begin{align*}
	(\forall n \in \mathbb{N})P(n).
\end{align*}

\vspace{\baselineskip}
\begin{proof}
	Since we have a statement we wish to prove the over the natural numbers, we use a proof by induction. This involves showing the base case and then the induction step.
	
	\vspace{\baselineskip}
	\begin{enumerate}[\hspace{\baselineskip}i.]
		\item \emph{Base Case.} We want to show $P(1)$ is true, which is easy to show.
		
		\begin{align*}
			\sum_{i = 1}^1 i = 1 = \frac{1 (1 + 1)}{2}
		\end{align*}
		
		\item \emph{Induction Step.} Assume $P(k)$ is true for some natural number $k$ so that we have
		
		\begin{align*}
			1 + 2 + \dots + k = \frac{k(k + 1)}{2}.
		\end{align*}
		
		We want to show that the statement $P(k + 1)$ must follow; which written out corresponds to
		
		\begin{align*}
			1 + 2 + \dots + (k + 1) = \frac{(k + 1)(k + 2)}{2}.
		\end{align*}
		
		Notice the similarity in structure on the right-hand side of both equations, which suggests a obvious substitution. We prove $P(k + 1)$ by starting with the right-hand side and deriving the left-hand side.
		
		\begin{align*}
			1 + 2 + \dots + (k + 1) &= \underbrace{1 + 2 + \dots + k}_\text{left-hand side of $P(k)$} + (k + 1) \\
			&= \frac{k (k + 1)}{2} + (k + 1) \\
			&= \frac{k (k + 1)}{2} + \frac{2 (k + 1)}{2} \\
			&= \frac{(k + 1)(k + 2)}{2}
		\end{align*}
		
		We are allowed to make the substitution in the second line, because we assumed this about the sum of the first $k$ natural numbers. We use this assumption to extend what we know to conclude the same thing about the first $(k + 1)$ natural numbers. Therefore, this concludes the proof; we have shown that $P(n)$ is true for all natural numbers $n$.
	\end{enumerate}
\end{proof}
\vspace{\baselineskip}

At this point, we should point out that never do we assume what we are trying to prove. Instead, in the induction step we assume $P(k)$ is true for one particular value of $k$ and show $P(k + 1)$. This assumption, that $P(k)$ is true, referred to as the \emph{induction hypothesis} in the inductive step. Our first inductive proof was rather verbose, but we now offer a more streamlined example to prove something similar.

\vspace{\baselineskip}
\begin{theorem}
	For all non-negative integers $n$, the summation formulas below have the following closed form solutions.
	
	\begin{enumerate}[\hspace{\baselineskip}i.]
		\item 
		\begin{align}
			\sum_{i = 1}^k i^2 &= \frac{k(k + 1)(2k + 1)}{6}
		\end{align}

		\item 
		\begin{align}
			\sum_{i = 1}^k i^2 &= \frac{k^2(k + 1)^2}{4}
		\end{align}
	\end{enumerate}
\end{theorem}
\begin{proof}
	We only prove the first statement by induction; the proof for the second statement is very similar.
	
	\vspace{\baselineskip}
	\begin{enumerate}[\hspace{\baselineskip}i.]
		\item \emph{Base Case.} We want to confirm the statement is true for $n = 1$.
		
		\begin{align*}
			\sum_{i = 1}^1 i^2 = 1 = \frac{1 (1 + 1) (2 + 1)}{6}
		\end{align*}
		
		\item \emph{Induction Step.} Assume that the statement is true for some arbitrary number $k$, so that
		
		\begin{align*}
			\sum_{i = 1}^k i^2 &= \frac{k(k + 1)(2k + 1)}{6}.
		\end{align*}
		
		We want to show that the statement is also true for $(k + 1)$. To show this, we write out the sum of the first $(k + 1)$ squares and then use the induction to simplify.
						
		\begin{align*}
			\sum_{i = 1}^{k + 1} i &= \underbrace{1^2 + 2^2 + \dots + k^2}_\text{induction hypothesis} + (k + 1)^2  \\
			&= \frac{k(k + 1)(2k + 1)}{6} + (k + 1)^2 \\
			&= \frac{(k + 1)(k + 2)(2k + 3)}{6}
		\end{align*}
		
		This is exactly what we needed to show, hence this concludes the proof by induction.
	\end{enumerate}
\end{proof}
\vspace{\baselineskip}


% ===========================================================
\section{Sequences and Recurrences}
% ===========================================================

In some sense, induction treats the natural numbers as an ordered list of elements, starting with 1 and increasing in the order we write them out.

\begin{align*}
	\mathbb{N} = \{ 1, 2, 3, \dots \}
\end{align*}

A natural generalization of this idea is to consider ordered lists of items with arbitrary elements, a notion we denote a \emph{sequence}, which can be represented as

\begin{align*}
	\langle a \rangle = a_1, a_2, a_3, \dots
\end{align*}

The subscript represents the \emph{term number}, which gives the position of a particular element, or \emph{term}, in the sequence. More formally a sequence is defined as a function from the natural numbers, which maps term numbers to terms.

\begin{align*}
	\begin{array}{rrrrrr}
		n & 1 & 2 & 3 & 4 & \dots \\
		& \downarrow & \downarrow & \downarrow & \downarrow & \\
		\langle a \rangle & a_1 & a_2 & a_3 & a_4 & \dots
	\end{array}
\end{align*}

Since the natural numbers are ordered, constructing this function ties the order of the natural numbers to those of the terms in the sequence, imposing an order. Two kinds of sequences that appear frequently in mathematics are \emph{arithmetic sequences} and \emph{geometric sequences}. The former is a sequence such that that the difference between consecutive terms is constant while the later is a sequence in which consecutive terms have the same ratio.

\begin{align*}
	\text{Arithmetic. } a, a + d, a + 2d, a + 3d, a + 4d, \dots \\
	\text{Geometric. } a, ar, ar^2, ar^3, ar^4, \dots \\
\end{align*}

In the above sequences, we can specify an arithmetic sequence completely by its first term $a$ and the common difference $d$. Meanwhile, for a geometric sequence, the first term $a$ is needed along with the common ratio $r$. By inspection, we can see that the $n$th term of an arithmetic sequence is

\begin{align*}
	a_n = a + (n - 1)d,
\end{align*}

while for a geometric sequence, we simply multiply the ratio to the first term $(n - 1)$ times, giving the formula

\begin{align*}
	b_n = a r^{n - 1}.
\end{align*}

Given the similarity in structure of sequences to the natural numbers, induction provides a natural way to prove things about these sequences. Consider the following theorem about the sum of first $n$ terms of arithmetic and geometric sequences.

\vspace{\baselineskip}
\begin{theorem}
	Consider the arithmetic sequence $\langle a \rangle$ and the geometric sequence $\langle b \rangle$, with the following terms
	
	\begin{align*}
		a_i &= a + (i - 1)d \\
		b_i &= a r^{i - 1}
	\end{align*}
	
	The sum of the first $n$ terms of each sequence have the following formula, given $r \neq 1$.
	
	\begin{align}
		\sum_{i = 1}^n a_i &=  a + (a + d) + (a + 2d) + \dots + (a + (n - 1)d) \\
		&= \frac{n}{2} \left( 2a + (n - 1)d \right)
	\end{align}
	
	\begin{align}
		\sum_{i = 1}^n b_i &=  a + ar + ar^2 + \dots + ar^{n - 1} \\
		&= a \frac{r^n - 1}{r - 1}
	\end{align}
\end{theorem}
\begin{proof}
	We prove both statements in order, using induction, starting with the sum of an arithmetic sequence.
	
	\vspace{\baselineskip}
	\begin{enumerate}[\hspace{\baselineskip}i.]
		\item \emph{Base Case.} This is for $n = 1$, because we are simply summing a sequence with 1 term.
		
		\begin{align*}
			\sum_{i = 1}^n a_i = a = \frac{1}{2} (2a + (1 - 1)d)
		\end{align*}
		
		\item \emph{Induction Step.} Assume this statement is true for the sum of the first $k$ terms of an arithmetic sequence. We show that also holds for the first $(k + 1)$ terms.
		
		\begin{align*}
			\sum_{i = 1}^{k + 1} a_i &=  a_{k + 1} + \sum_{i = 1}^k a_i \\
			&= a_{k + 1} + \frac{k}{2} \left( 2a + (k - 1)d \right) \\
			&= (a + kd) + \frac{k}{2} \left( 2a + (k - 1)d \right) \\
			&= \frac{k}{2} \left( \frac{2a}{k} + 2d \right) + \frac{k}{2} \left( 2a + (k - 1)d \right) \\
			&= \frac{k}{2} \left[ 2a \frac{k + 1}{k} + (k + 1)d
			 \right] \\
			&= \frac{k + 1}{2} \left( 2a + kd \right)
		\end{align*}
	\end{enumerate}
	\vspace{\baselineskip}
	
	This proves the formula for the sum for the first $n$ terms of an arithmetic sequence; we now approach a similar proof for the sum of the first $n$ terms of a geometric sequence.
	
	\vspace{\baselineskip}
	\begin{enumerate}[\hspace{\baselineskip}i.]
		\item \emph{Base Case.} Once again, this is trivial for $n = 1$ since that involves summing a one-term sequence.
		
		\begin{align*}
			\sum_{i = 1}^n b_i = a = a \frac{r^1 - 1}{r - 1}
		\end{align*}
		
		\item \emph{Induction Step.} Assume this statement is true for the sum of the first $k$ terms. We show that also holds for the first $(k + 1)$ terms.
		
		\begin{align*}
			\sum_{i = 1}^{k + 1} b_i &=  b_{k + 1} + \sum_{i = 1}^k a_i \\
			&= b_{k + 1} + a \frac{r^k - 1}{r - 1} \\
			&= ar^k +  a \frac{r^k - 1}{r - 1} \\
			&= a \frac{r^k(r - 1)}{r - 1} + a \frac{r^k - 1}{r - 1} \\
			&= a \frac{r^k(r - 1) + r^k - 1}{r - 1} \\
			&= a \frac{r^{k + 1} - 1}{r - 1}
		\end{align*}
	\end{enumerate}	
\end{proof}
\vspace{\baselineskip}

Notice that the algebra involved in these proofs were much messier that for those we dealt with before, yet we forged ahead. One benefit of induction is that that if the statement is true, then the algebra will almost certainly work out, so we know how exactly to simply the expression. Now consider a different kind of sequence.

\vspace{\baselineskip}
\begin{theorem}
	If we define a sequence $\langle a \rangle$ recursively, so that
	
	\begin{align*}
		a_n =
		\begin{cases}
			1 &: n = 1 \\
			a_{n - 1} + n &: n > 1
		\end{cases}
	\end{align*}
	
	then, for all natural numbers $n$, this sequence is equivalent to the polynomial
	
	\begin{align}
		a_n = \frac{n(n + 1)}{2}.
	\end{align}
\end{theorem}
\begin{proof}
	We provide an informal proof of this statement, based on our previous result; a more thorough proof would simplify formalize this with induction. Consider $a_n$ for some arbitrary $n$, we begin simplifying the expression using the second rule.
	
	\begin{align*}
		a_n &= a_{n - 1} + n \\
		&= a_{n - 2} + (n - 1) + n \\
		&= a_{n - 3} + (n - 2) + (n - 1) + n \\
		&= a_1 + 2 + \dots + (n - 2) + (n - 1) + n \\
		&= 1 + 2 + \dots + (n - 1) + n \\
		&= \frac{n (n + 1)}{2}
	\end{align*}
	
	Notice there is however, one additional case we need to check now, which is when $n = 1$, but clearly the formula trivially holds here.
\end{proof}
\vspace{\baselineskip}

The sequence we encountered above is rather different from those which we previously encountered. Such a sequence is by a \emph{recurrence relation} is characterized by two features:

\begin{enumerate}[\hspace{\baselineskip}i.]
	\item \emph{Base cases.} Explicit values of the sequence for a small number of terms. In this case, by definition, $a_1 = 1$.
	\item \emph{Recursive definition.} A relationship between the value of the sequence $a_n$ to other values, typically smaller. In the above example, for non-base cases, we were given
	
	\begin{align*}
		a_n = a_{n - 1} + n.
	\end{align*}
\end{enumerate}

While this type of recurrence equation describes a perfectly well-defined sequence, a more explicit definition of each term is often desirable. This explicit formula is called \emph{closed-form solution} of the recurrence, which usually relates $a_n$ directly to $n$, rather than other terms. However the recursive definition is sometimess cleaner; we can convey the essence of an arithmetic and geometric sequence with the following recurrence relations.

\begin{align*}
	\text{Arithmetic. } a_n = \begin{cases}
		a &: n = 1 \\
		a_{n - 1} + d &: n > 1
	\end{cases} 
\end{align*}
\begin{align*}
	\text{Geometric. } b_n = \begin{cases}
 		a &: n = 1 \\
 		b_{n - 1} \cdot r &: n > 1
	\end{cases}
\end{align*}

The closed forms of these are the term expression we began with, which are

\begin{align*}
	a_n &= a + (n - 1)d \\
	b_n &= a r^{n - 1}
\end{align*}

% ===========================================================
\section{Extending Induction}
% ===========================================================

So far the power of induction seems rather limited since it applies to only statements of a very particular form, namely statements which are true over the natural numbers. However it turns out that induction can can applied to prove statements which are true over an entire class of sets, including for all non-negative integers, or all even numbers, or all odd numbers, which is explained with the following theorem.

\vspace{\baselineskip}
\begin{theorem}
	Suppose we want to prove some predicate $P(n)$ for all $n \in A$, where $A$ is some set which can be expressed as
	
	\begin{align*}
		A = \{ a_1, a_2, a_3, \dots \}.
	\end{align*}
	
	A more precise condition is that $A$ be countable, but we return to this notion in a later chapter. Then it is sufficient to show
	\vspace{\baselineskip}
	\begin{enumerate}[\hspace{\baselineskip}i.]
		\item \emph{Base Case.} $P(a_1)$ is true.
		\item \emph{Induction Step.} if $P(a_k)$ is true for any natural number $k$, then $P(a_{k + 1})$ must be true
	\end{enumerate}
\end{theorem}
\begin{proof}
	We prove this by reduction, showing that this technique is exactly identical to induction. First consider defining the following new predicate which is related to $P(n)$.
	
	\begin{align*}
		Q(n) \equiv P(a_n)
	\end{align*}
	
	First notice that the elements of $A$ are indexed by $n$, so to prove $P(a_n)$ for all $a_n \in A$ is equivalent to proving $Q(n)$ for all $n \in \mathbb{N}$. At this point, we notice that we can apply induction on $n$ to prove $Q(n)$ for all natural numbers, which entails showing the following two things.
	
	\vspace{\baselineskip}
	\begin{enumerate}[\hspace{\baselineskip}i.]
		\item \emph{Base Case.} $Q(1)$ is true.
		\item \emph{Induction Step.} if $Q(k)$ is true for any natural number $k$, then $Q(k + 1)$ must be true
	\end{enumerate}
	\vspace{\baselineskip}
	
	Now if we show these both hold, then this proves $Q(n)$ over the natural numbers, which is equivalent to showing $P(a)$ is true for all $a$ in $A$. All that is left is to translate these conditions back into terms of $P$.
	
	\vspace{\baselineskip}
	\begin{enumerate}[\hspace{\baselineskip}i.]
		\item \emph{Base Case.} $P(a_1)$ is true.
		\item \emph{Induction Step.} if $P(a_k)$ is true for any natural number $k$, then $P(a_{k + 1})$ must be true
	\end{enumerate}
\end{proof}
\vspace{\baselineskip}

This shows that to prove some predicate over a set which can be expressed as an ordered list, we can: (1) show that it is true for the first element, and (2) that if it is true for some element, it is true for the next element.

\begin{align*}
	\begin{array}{l}
		P(a_1) \text{ is true} \\
		P(a_1) \implies P(a_2) \implies P(a_3) \implies P(a_4) \implies \dots \\
		\hline
		P(a_n) \text{ is true for all } n \in A
	\end{array}
\end{align*}

This suggests the common ``domino'' analogy for understanding induction. To prove that that a line of dominoes have fallen down, it is sufficient to show that the first one had fallen down and that if a domino has been knocked down, it causes the next one to fall.

With this theorem, we can introduce some common patterns used in induction proofs to prove statements over different sets, and armed with these we can revisit and generalize some of our past results.


\vspace{\baselineskip}
\begin{center}
	\begin{tabular}{ccc}
		\toprule
		Set & Base Case & Induction Step \\
		\midrule
		$\mathbb{N} = \{ 1, 2, 3, \dots \}$ & $P(1)$ & $P(k) \implies P(k + 1)$ \\
		$\mathbb{N}_0 = \{ 0, 1, 2, \dots \}$ & $P(0)$ & $P(k) \implies P(k + 1)$ \\
		$\{ 0, 2, 4, \dots \}$ & $P(2)$ & $P(k) \implies P(k + 2)$ \\
		$\{ 1, 3, 5, \dots \}$ & $P(1)$ & $P(k) \implies P(k + 2)$ \\
		$\{ 2^0, 2^1, 2^2, \dots \}$ & $P(1)$ & $P(k) \implies P(2k)$ \\
		$\{ a, a + d, a + 2d, \dots \}$ & $P(a)$ & $P(k) \implies P(k + d)$ \\
		\bottomrule
	\end{tabular}
\end{center}
\vspace{\baselineskip}

So, for example, to prove a statement is true for all odd natural number, we first show that it is true for 1, and that if it true for some odd integer $k$ it is also true for $k + 2$. This is a direct application of the theorem we proved where we define

\begin{align*}
	a_k = 2k - 1, \text{ so } A = \{1, 3, 5, \dots \}
\end{align*}

Now, with this technique, we can extend many of the inequalities we showed previously, and solve a more general recurrence.

\vspace{\baselineskip}
\begin{theorem}[Generalized Triangle Inequality]
	For any number of real numbers $n \ge 2$, the absolute value of their sum is bounded by the sum of their absolute values.
	
	\begin{align*}
		|a_1 + a_2 + \dots + a_n| \le |a_1| + |a_2| + \dots + |a_n|
	\end{align*}
\end{theorem}
\begin{proof}
	We apply a proof by induction on $n$ over the set of natural numbers greater than 1. Our base case should then be $n = 2$.
	
	\vspace{\baselineskip}
	\begin{enumerate}[\hspace{\baselineskip}i.]
		\item \emph{Base Case.} In the case where $n = 2$, this reduces into the triangle inequality we proved previously.
		
		\begin{align*}
			|a_1 + a_2| \le |a_1| + |a_2|
		\end{align*}
		
		\item \emph{Induction Step.} Assume that this generalized triangle inequality holds for $k$ numbers, so that
		
		\begin{align*}
			|a_1 + a_2 + \dots + a_k| \le |a_1| + |a_2| + \dots + |a_k|.
		\end{align*}
		
		 We will now show that this also holds for $k + 1$ numbers;  consider the absolute value of their sum.
		 
		 \begin{align*}
		 	|a_1 + \dots + a_k + a_{k + 1}| &= |(a_1 + \dots + a_k) + a_{k + 1}| \\
		 	&\le |a_1 + \dots + a_k| + |a_{k + 1}| \\
		 	&\le |a_1| + |a_2| + \dots + |a_k| + |a_{k + 1}| \\
		 \end{align*}
		 
		 In the last step, we make the substitution from the induction hypothesis, which complete the inductive step and thus the proof.
	\end{enumerate}
\end{proof}
\vspace{\baselineskip}

We can now approach more general solutions recurrence relations as well. A sequence of the form

\begin{align*}
	x_n =
	\begin{cases}
		x_1 &: n = 1 \\
		a x_{n - 1} + b &: n > 1
	\end{cases}
\end{align*}

is called a first-order linear recurrence relation, which is a well-defined sequence, given the base case $x_n$ is defined. The ``first-order'' refers to the fact that $x_n$ refers only to one previous term, and the ``linear'' comes from the fact that $x_n$ is a linear function of $x_{n - 1}$, plus some constant.

\vspace{\baselineskip}
\begin{theorem}[First-Order Linear Recurrences]
	The solution to a first-order linear recurrence relation defined as
	
	\begin{align*}
		x_n =
		\begin{cases}
			x_1 &: n = 1 \\
			a x_{n - 1} + b &: n > 1
		\end{cases}
	\end{align*}
	
	has closed-form solution
	
	\begin{align*}
		x_n = a^{n - 1} x_1 + b \frac{a^{n - 1} - 1}{a - 1}, & \hspace{\baselineskip} a \neq 1 \\
		x_n = x_1 + (n - 1)b, &\hspace{\baselineskip} a =  1
	\end{align*}
\end{theorem}
\begin{proof}
	Unsurprisingly, we are going to prove this by induction in a long and messy proof. However, we actually have two cases to consider, one where $a$ is 1 and one where it is not. We will first prove the more general of the two results, which is when $a \neq 1$.
	
	\begin{enumerate}[\hspace{\baselineskip}i.]
		\item \emph{Base Case.} This is easy to check, since it only involves the first erm, which we are given.
		
		\begin{align*}
			x_1 =  a^0 x_1 + b \frac{a^0 - 1}{a - 1}
		\end{align*}
		\item \emph{Induction Step.} Assume that the closed form is correct for the $k$th term so that
		
		\begin{align*}
			x_k = a^{k - 1} x_1 + b \frac{a^{k - 1} - 1}{a - 1}
		\end{align*}
		
		Now we show the formula also holds for the next term, namely for the $(k + 1)$th term. Do this this, we apply the recursive definition of this term.
		
		\begin{align*}
			x_{k + 1} &= a x_k + b \\
			&= a \left( a^{k - 1} x_1 + b \frac{a^{k - 1} - 1}{a - 1} \right) + b \\
			&= a^k x_1 + ab \frac{a^{k - 1} - 1}{a - 1} + b \\
			&= a^k x_1 + b \left( a \frac{a^{k - 1} - 1}{a - 1} + 1 \right) \\
			&= a^k x_1 + b \left( \frac{a^k - a}{a - 1} + \frac{a - 1}{a - 1} \right) \\
			&= a^k x_1 + b \frac{a^k - 1}{a - 1}
		\end{align*}
	\end{enumerate}
	
	Now, unfortunately, we still have to deal with the case of $a = 1$ separately. In this case, however, this recurrence is easier to work with.
	
	\begin{align*}
		x_n =
		\begin{cases}
			x_1 &: n = 1 \\
			x_{n - 1} + b &: n > 1
		\end{cases}
	\end{align*}
	
	\begin{enumerate}[\hspace{\baselineskip}i.]
		\item \emph{Base Case.} Once again, it is easy to verify the closed-form solution works for just one term.
		
		\begin{align*}
			x_1 = x_1 + (1 - 1)b
		\end{align*}
		
		\item \emph{Induction Step.} Assume the formula holds for the $k$th term, and show that it also hold for the term after that.
		
		\begin{align*}
			x_{k + 1} &= x_{k} + d \\
			&= x_1 + (k - 1)d + d \\
			&= x_1 + kd
		\end{align*}
	\end{enumerate}
\end{proof}
\vspace{\baselineskip}

Notice that this immediately proves the equivalence of the recurrence relation definition of arithmetic and geometric sequences to the one we originally used. If we take $a = 1$, then we get

\begin{align*}
	x_n =
	\begin{cases}
		x_1 &: n = 1 \\
		x_{n - 1} + b &: n > 1
	\end{cases} \\
	\text{then } x_n = x_1 + (n - 1) d
\end{align*}

which refers to arithmetic sequences, while if we take $b = 0$ then we find

\begin{align*}
	x_n =
	\begin{cases}
		x_1 &: n = 1 \\
		ax_{n - 1} &: n > 1
	\end{cases} \\
	\text{then } x_n = x_1 a^{n - 1}
\end{align*}

which corresponds exactly to geometric sequences. Of course, it may seem odd that we need a separate case at all to account for when $a = 1$, but it turns out, in some sense, the general formula still holds. Rather than being thrown off guard by noticing the $(a - 1)$ in the denominator, we can instead take the limit as $a$ tends to 1.

\begin{align*}
	\lim_{a \rightarrow 1} x_n &= \lim_{a \rightarrow 1} \left( a^{n - 1} x_1 + b \frac{a^{n - 1} - 1}{a - 1} \right) \\
	&= x_1 \lim_{a \rightarrow 1} a^{n - 1} + b \lim_{a \rightarrow 1} \frac{a^{n - 1} - 1}{a - 1} \\
	&= x_1 + (n -1) b
\end{align*}

The first limit should be easy to see; the second one is the difference quotient representing derivative of the function

\begin{align*}
	f(a) = a^{n - 1}
\end{align*}

at $a = 1$. This is out of the scope of this chapter, but it goes to show that this closed-form solution is not as disjoint as it may appear to be.


% ===========================================================
\section{Strong Induction}
% ===========================================================


The power of induction is that it allow use to make stronger assumptions in the process of proving our statement. Instead of proving the statement directly for any arbitrary element $n \in \mathbb{N}$, in the induction step, we assume $P(k)$ is true for some natural number $k$ and simply have to prove $P(k + 1)$ to be true. This advantage comes from knowing something about the structure of the natural numbers themselves.

However, it turns out that we are allowed to assume something even stronger during the induction step.

\vspace{\baselineskip}
\begin{theorem}[Principle of Strong Induction]
	Let $P(n)$ be a predicate defined for natural numbers $n \in \mathbb{N}$. If the following two statements hold,
	
	\vspace{\baselineskip}
	\begin{enumerate}[\hspace{\baselineskip}i.]
		\item $P(1)$ is true
		\item if $P(1), P(2), \dots, P(k)$ are true, then $P(k + 1)$ is true
	\end{enumerate}
	\vspace{\baselineskip}

	then $P(n)$ is true for all natural numbers $n \in \mathbb{N}$.
\end{theorem}
\begin{proof}
	We prove this by showing that this is essential equivalent to the principle of induction. First, we define the following predicate
	
	\begin{align*}
		Q(n) \equiv P(1) \land P(2) \land \dots \land P(n)
	\end{align*}
	
	so that $Q(n)$ is true if and only if $P(i)$ is true for all $i$ up to $n$. Notice showing $P(n)$ is true for all $n \in \mathbb{N}$ is equivalent to showing $Q(n)$ is true for all natural numbers. Remember the principle of mathematical induction tell us to show this predicate is true, all we need is to show is the following.

	\vspace{\baselineskip}
	\begin{enumerate}[\hspace{\baselineskip}i.]
		\item $Q(1)$ is true
		\item if $Q(k)$ is true, then $Q(k + 1)$ is true
	\end{enumerate}
	\vspace{\baselineskip}

	Therefore, if we show these two statements hold, then $P(n)$ must be true as well. First consider the first statement; by definition, $Q(1)$ is the same as $P(1)$. Next, we consider the implication, which we express in terms of $P$.
	
	\begin{align*}
		Q(k) &\implies Q(k + 1) \\
		P(1) \land \dots \land P(k) &\implies \underbrace{P(1) \land \dots \land P(k)}_\text{redundant} \land P(k + 1) \\
		P(1) \land \dots \land P(k) &\implies P(k + 1)
	\end{align*}
	
	Therefore, we have shown to prove $Q(n)$ is true far all natural numbers $n$, and thus $P(n)$ is true, we can show
	
	\vspace{\baselineskip}
	\begin{enumerate}[\hspace{\baselineskip}i.]
		\item $P(1)$ is true, and
		\item if $P(1), P(2), \dots, P(k)$ are true, then $P(k + 1)$ is true.
	\end{enumerate}
\end{proof}
\vspace{\baselineskip}

This allows us to augment the amount of information we get for ``free'' during the induction step, which allows us to prove statements that we otherwise might not be able to. While this unlocks the door to more complicated recurrence relations, we can also re-visit a fact about the natural numbers we have been relying on without proof.

\vspace{\baselineskip}
\begin{theorem}[Prime Factorization]
	Any natural number $n$ can be expressed as a, possibly empty, product of primes; meaning
	
	\begin{align*}
		n = \prod_{i} p_i^{\alpha_i}
	\end{align*}
	
	for some set of primes $\{ p_i \}$ and corresponding powers $\{ \alpha_i \}$. By convention, we say that the empty product is 1. In other words, every natural number has a prime factorization.
\end{theorem}
\begin{proof}
	We can prove this by strong induction on $n$, with a base case of $n = 1$.
	
	\vspace{\baselineskip}
	\begin{enumerate}[\hspace{\baselineskip}i.]
		\item \emph{Base case.} As explained, the empty product is taken to be 1 so 1 is trivially the product of primes, namely none at all.
		
		\item \emph{Induction Step.} Assume that this holds for all numbers less than or equal to some $k$. We now prove this for $(k + 1)$ by casework. Now by definition, every number greater than or equal to 2 is either prime or composite. If $(k + 1)$, then we are immediately done since we take
		
		\begin{align*}
			k + 1 = \underbrace{(k + 1)^1}_\text{prime}
		\end{align*}
		
		Otherwise, it it composite, which means by definition, it can be written as the product of two numbers less than itself, since it has a divisor that is not equal to 1 or itself. Therefore,
		
		\begin{align*}
			k + 1 = a \cdot b, \text{ for some } a, b \le k.
		\end{align*}
		
		By our induction hypothesis, we know both $a$ and $b$ can be expressed as a product of primes. Since $(k + 1)$ is the product of these, it can clearly also be expressed as a product of primes.
	\end{enumerate}
\end{proof}
\vspace{\baselineskip}

It turns out, as you probably already know, that something even stronger is true--that every natural number has a unique prime factorization. This is a theorem called the fundamental theorem of arithmetic, which is currently outside of our grasp, but which we will prove later. As we will see, what we proved just now is an important building block in the proof of the fundamental theorem. A proof that follows a similar pattern, and is also foundational to computers today, is the following.

\vspace{\baselineskip}
\begin{theorem}
	Every non-negative integer can be expressed in binary so that it is a sum of distinct powers of 2. In other words, there exists a subset of the non-negative integer $B \in \mathbb{N}_0$ such that
	
	\begin{align*}
		n = \sum_{i \in B} 2^i
	\end{align*}
	
	Once again, we use the convention that the empty sum is 0, which allows us to include 0 in this theorem.
\end{theorem}
\begin{proof}
	We will once again use strong induction.
	
	\vspace{\baselineskip}
	\begin{enumerate}[\hspace{\baselineskip}i.]
		\item \emph{Base case.} Once again, we trivially take $B = \{ \}$ represent 0. The reason this convention makes sense is because we do the same in the decimal numbers system; while we can express 
		
		\begin{align*}
			15 = 1 \times 10^1 + 5 \times 10^0,
		\end{align*}
		
		we can only express 0 by letting all the coefficients of powers of ten be 0. The same reasoning motivates our choice here.
		
		\item \emph{Induction step.} Assume that all non-negative numbers less than or equal to $k$ can be expressed as the sum of distinct powers of 2. We will now show the same holds for $(k + 1)$. Now we have two cases to consider, one where this number is even and one where this is odd. In the former we can write
		
		\begin{align*}
			k + 1 &= 2m, m \in \mathbb{N}_0
		\end{align*}
		
		Since $m$ must be strictly less than $(k + 1)$, from the induction hypothesis, we know that it can be expressed as a sum of distinct powers of two. Clearly, when we multiply each of these by two (raising their exponents by one), we find a new representation. The second case to consider is when
		
		\begin{align*}
			k + 1 = k + 2^0
		\end{align*}
		
		is odd. This means that $k$ is even and by the induction hypothesis can be represented as a sum of unique powers of 2. Since it is even, this sum cannot include $2^0$, which means we simply add this to find the desired representation.
	\end{enumerate}
\end{proof}
\vspace{\baselineskip}

Having shown some unique applications of this proof technique, we return, as promised, to more complicated recurrence relations. In particular, we can now prove closed form representations for recurrences beyond first order, in which each term is a function of more than just the previous term. In the extreme case, we could have a sequence for which each term is defined in terms of all its predecessors.

\begin{align*}
	a_n =
	\begin{cases}
		a_1 &: n = 1 \\
		f(a_{n - 1}, a_{n - 2}, \dots, a_1) &: n > 1
	\end{cases}
\end{align*}

However, more often, we run into cases where each term is a function of two, three or some other finite number of terms. A famous such sequence is the \emph{Fibonacci sequence},

\begin{align*}
	F_n =
	\begin{cases}
		1 &: n = 1 \\
		1 &: n = 2 \\
		F_{n - 1} + F_{n - 2} &: n > 2
	\end{cases}
\end{align*}
 
in which the first two terms are 1 and each term after this is the sum of the previous two. The first few terms of this sequence are

\begin{align*}
	\langle F \rangle = 1, 1, 2, 3, 5, 8, 13, 21, 35, 56, 91, \dots
\end{align*}

It turns out that there indeed is a closed-form solution to the Fibonacci sequence, albeit one which seems far from intuitive, which we will show below 

\vspace{\baselineskip}
\begin{theorem}
	The closed form of the Fibonacci sequence is given by the exact formula,
	
	\begin{align*}
		F_n = \frac{1}{\sqrt{5}}\left( \varphi^n - \psi^n \right)
	\end{align*}
	
	where $\varphi$ and $\psi$ are the two roots of the quadratic $x^2 - x - 1$, or explicitly
	
	\begin{align*}
		\varphi = \frac{1 + \sqrt{5}}{2}, \, \psi = \frac{1 - \sqrt{5}}{2}.
	\end{align*}
	
	The number $\varphi$ appears in other places and is often referred to as the golden ratio.
\end{theorem}
\begin{proof}
	It may not immediately be obvious that the closed firm solution holds; it seems rather miraculous that this formula, peppered with radicals, should somehow yield whole numbers, and additionally correspond to the Fibonacci sequence. Yet, we do just that.
	
	\vspace{\baselineskip}
	\begin{enumerate}[\hspace{\baselineskip}i.]
		\item \emph{Base Cases.}
		The recursive part of the Fibonacci sequence definition only applies to terms past the fist two, hence we will have to show the formula hold manually; this is will be our base cases.
		
		\begin{align*}
			F_1 &= \frac{1}{\sqrt{5}}\left( \varphi^1 - \psi^1 \right) \\
			&= \frac{1}{\sqrt{5}}\left( \frac{1 + \sqrt{5}}{2} - \frac{1 - \sqrt{5}}{2} \right) \\
			&= \frac{1}{\sqrt{5}} \frac{2\sqrt{5}}{2} \\
			&= 1
		\end{align*}
		
		To show the formula also holds for the seconds term, we use a trick, which is noting that since $\phi$ and $\psi$ are roots of $x^2 - x - 1$,
		
		\begin{align*}
			\varphi^2 - \varphi - 1 = 0, \text{ so } \varphi^2 = \varphi + 1 \text{ and} \\
			\psi^2 - \psi - 1 = 0, \text{ so } \psi^2 = \psi + 1. \\
		\end{align*}
		
		We can use this to simply the exponents in the canonical closed form representation of the second term.
		
		\begin{align*}
			F_2 &= \frac{1}{\sqrt{5}}\left( \varphi^2 - \psi^2 \right) \\
			&= \frac{1}{\sqrt{5}}\left( \varphi + 1 - \psi - 1 \right) \\
			&= \frac{1}{\sqrt{5}}\left( \varphi - \psi \right) \\
			&= F_1 \\
			&= 1
		\end{align*}
		
		This matches with the first few terms of the first two terms of the Fibonacci sequence, which means the base cases hold.
		
		\item \emph{Induction Step.} Assume that the formula holds for all terms with number less than or equal to $k$, we want to show that it will also hold for the next term. Specifically, we will use,
		
		\begin{align*}
			F_k = \frac{1}{\sqrt{5}} \left( \varphi^k - \psi^k \right) \\
			F_{k - 1} \frac{1}{\sqrt{5}} \left( \varphi^{k - 1} - \psi^{k - 1} \right).
		\end{align*}
		
		We can now plug this into the recursive definition of $F_{k + 1}$ to show that the closed form also holds for the next term.
		
		\begin{align*}
			F_{k + 1} &= F_k + F_{k - 1} \\
			&= \frac{1}{\sqrt{5}} \left( \varphi^k - \psi^k \right) +\frac{1}{\sqrt{5}} \left( \varphi^{k - 1} - \psi^{k - 1} \right) \\
			&= \frac{1}{\sqrt{5}} \left( \varphi^k + \varphi^{k - 1} - \psi^k - \psi^{k - 1} \right) \\
			&= \frac{1}{\sqrt{5}} \left( \varphi^{k - 1} (\varphi + 1) - \psi^{k - 1} (\psi + 1) \right)
		\end{align*}
		
		Now we can use the observation we used earlier, in reverse, to swiftly finish the proof. Namely $x + 1$ equals $x^2$ when $x$ is $\varphi$ or $\psi$. Making the substitution, we immediately get what we want.
		
		\begin{align*}
			F_{k + 1} &= \frac{1}{\sqrt{5}} \left( \varphi^{k - 1} (\varphi + 1) - \psi^{k - 1} (\psi + 1) \right) \\
			&= \frac{1}{\sqrt{5}} \left( \varphi^{k - 1} \varphi^2 - \psi^{k - 1} \psi^2 \right) \\
			&= \frac{1}{\sqrt{5}} \left( \varphi^{k + 1} - \psi^{k + 1} \right)
		\end{align*}
	\end{enumerate}
	This is what we needed to show to prove our claim by induction.
\end{proof}
\vspace{\baselineskip}

% ===========================================================
\section{The Well-Ordering Principle}
% ===========================================================

We can now show another equivalent fact about the natural numbers which is inextricably linked to induction. This is called the well-ordering principle, which states that every non-empty subset of the natural numbers has a smallest element.

\vspace{\baselineskip}
\begin{theorem}[Well-ordering Principle]
	Let $A$ be a non-empty subset of the natural numbers, $A \subseteq \mathbb{N}$. $A$, then, must have a smallest element.
\end{theorem}
\begin{proof}
	We instead prove the contrapositive of this statement, that if $A$ is a subset of the natural numbers which has no smallest element, then it is the empty set. Now consider the relative complement of $A$ in the natural numbers, which we denote
	
	\begin{align*}
		B = \mathbb{N} - A
	\end{align*}
	
	Notice 1 is an element of $B$, since 1 cannot be an element of $A$, otherwise this would be its least element. Now assume
	
	\begin{align*}
		1, 2, \dots, k \in B,
	\end{align*}
	
	which means that none of these numbers are elements of $A$. This implies that $(k + 1)$ cannot be in $A$ otherwise it would be its smallest element. Therefore we have shown that $(k + 1)$ is in $B$. Therefore, we have
	
	\vspace{\baselineskip}
	\begin{enumerate}[\hspace{\baselineskip}i.]
		\item $1 \in B$, and
		\item if $1, 2, \dots, k \in B$, then $(k + 1) \in B$.
	\end{enumerate}
	\vspace{\baselineskip}
	
	Hence from the principle of strong induction, we know that that $B$ contains the natural numbers. However, since $B$ is the result of subtracting $A$ from the natural numbers, we also know that $B$ is a subset of $\mathbb{N}$, which together implies
	
	\begin{align*}
		B = \mathbb{N}. \\
		\therefore A &= \mathbb{N} - B \\
		&= \phi
	\end{align*}
	
	This completes the proof of the contrapositive, which means that if $A$ has no smallest element, then it must be the empty set. Equivalently, then, every non-empty subset of the natural numbers must have a smallest element. 
\end{proof}
\vspace{\baselineskip}

In some sense, the well-ordering principle is the opposite of induction, which makes it a powerful tool for proof by contradiction over the natural numbers. If a predicate $P(n)$ about all natural numbers is false then the set 

\begin{align*}
	\{ n \in \mathbb{N} : P(n) \text{ is false}\}
\end{align*}

will be non-empty subset of the natural numbers. It must be non-empty because otherwise this means the predicate is true for all natural numbers. By the well-ordering principle, this set must have a smallest element. A common pattern in a proof by contradiction is the following:

\vspace{\baselineskip}
\begin{enumerate}[\hspace{\baselineskip}i.]
	\item Assume that the proposition is false;
	\item By the well-ordering principle, there must be a smallest natural number for which the predicate is false, denote this $k$;
	\item Show that we can somehow derive a smaller number for which this proposition is also false, contradicting the original assumption that $k$ was the smallest such couter-example.
\end{enumerate}